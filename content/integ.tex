\paragraph{Euclidean division}
In the case of natural numbers, a natural number can be expressed as a product of any two natural numbers and a remainder.
Examples are shown below:
\begin{align*}
    35 &= 5(7) + 0 \\
    123 &= 6(20) + 3
\end{align*}

\subsection{Divisors and Multiples}
\paragraph{Divisors}
A divisor of $n \in \natl$ is a number $k \in \natl$ such that $\frac{n}{k} \in \natl$, i.e. the quotient between the dividend and divisor is a natural number.
Divisors can also be called the \emph{factors} of a natural number.

\subparagraph{Proper divisors}
A proper divisor of a number is a divisor that is \emph{not equal} to the number.

\subparagraph{Greatest Common Factor/Divisor}
The greatest common factor or divisor (GCD) of a set of natural numbers is the greatest natural number, not necessarily an element of the set, that divides all of the numbers in the set.
Formally:
\[
GCD(S \subset \natl) = x \in \natl\ s.t.\ x \mid a\ \forall\ a \in S
\]

\paragraph{Multiples}
A multiple of $n \in \natl$ is a number $M \in \natl$ such that $\exists\ k \in \natl : M = kn$ i.e. there exists a natural number $k$ where $M$ is a product of $k$ and $n$.

\subparagraph{Least Common Multiple}
The least common multiple (LCM) of a set of natural numbers is the least natural number, not necessarily an element of the set, such that every element in the set divides the number.
Formally:
\[
LCM(S \subset \natl) = x \in \natl\ s.t.\ a \mid x\ \forall\ a \in S
\]

These definitions imply that:
\begin{enumerate}
    \item A natural number $n$ is a divisor \emph{and} a multiple but \emph{not} a proper divisor of itself.
    \item If $k \mid n$ then $k$ is a divisor of $n$ and $n$ is a divisor of $k$.
\end{enumerate}

\subsection{Euclidean Algorithm}
The Euclidean algorithm is a fast \emph{recursive} method to calculate the GCF of a two numbers.
Reduction algorithms can be applied to a list of numbers based on the algorithm.
The algorithm assumes that the GCF of two numbers is the same as the GCF of one of the numbers and the difference of the numbers.
Formally, with $x > y$:
\[
GCD(x, y) = GCD(x-y, y) = GCD(x-y, x)
\]

\subsection{Applications of the properties of integers}
\paragraph{Fundamental theorem of arithmetic}
The theorem states that natural numbers other than $1$ are either \emph{prime numbers} or can be uniquely represented as \emph{products} of prime numbers.

\paragraph{Number of divisors of a number}
From the fundamental theorem of arithmetic and counting techniques (see section \ref{stat}), it is possible to determine the number of factors of an integer.
As examples:

\begin{align*}
100 &= 2^2 \cdot 5^2 \\
45 &= 3^2 \cdot 5^1
\end{align*}

The first number ($100$) has $(2+1)(2+1) = 9$ factors.
The second number ($45$) has $(2+1)(1+1) = 6$ factors.

\paragraph{Number of ``trailing zeros" in a factorial}
In finding the number of trailing zeros in a factorial $m$, it is necessary to find the largest power of $10$ that can divide the factorial expression.
Since $10 = 2\cdot 5$ we find $min(f(2),\ f(5))$ where $f(n\in P)$ gives the exponent of the prime number $n$ in the prime factorization of the factorial.
The value of $f(n)$ can be obtained from continuous division of $n$ from the operand $m$.

\paragraph{Modulo Congruence}

\paragraph{Chinese Remainder Theorem}
