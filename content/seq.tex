\subsection{Sequences and their sums}

\paragraph{Sequences}
A sequence is a properly defined collection of symbols or numbers.
Real sequences $\{a_n\}$ where $a_i \in \real$ can be defined in terms of the progression of the terms in them.
Examples of real sequences are:
\begin{enumerate}
  \item $\{ a_n\} = \{ 23, 24, 25, 26, 27, 28\}; n = 6$
  \item $\{ g_n\} = \{ -2, 4, -8, 16, -32, 64\}; n = 6$
  \item $\{ t_n\} = \{ 1, 1, 2, 3, 5, 8, 13, 21, 34}; n = 9$
  \item $\{ f_n\} = \{ 1, 3, 6, 10, 15, 21, 28\}; n = 7$
\end{enumerate}

\paragraph{Series}
A series is defined as the sum of terms of a given sequence.

\subparagraph{Convergence and divergence of infinite series}
An infinite series is said to be \emph{converging} if the limit to infinity of a sum can be defined.
Otherwise, it is said to be divergent.

\subsubsection{Arithemetic progressions and geometric progressions}

\paragraph{Arithmetic sequences} 
Arithmetic sequences are sequences of real numbers wherein $\exists\ d \in \real a_n = a_{n-1} + d$.
This means that consecutive elements of the sequence are separated by a \emph{common difference} $d$.
The first example given above is an arithmetic sequence with $d = 1$ and $n = 6$.
Arithmetic sequences can only be [strictly] \emph{increasing} or \emph{decreasing} depending on the sign of $d$.
If $d > 0$, the sequence is increasing, else if $d < 0$, the sequence is decreasing.

\paragraph{Arithemetic series}
Generally the series sums are expressed as the following:

\[
  S_a(n) = \dfrac{a_1 + a_n}{2}\cdot n = \dfrac{2a_1 + d(n-1)}{2}\cdot n
\]

where $n$ is the number of terms, $a_1$ is the first term, $a_n$ is the last term, and $d$ is the common difference.

\subparagraph{Divergence of series}
Generally, the series can be expressed in terms of the first term.
Given the equation above, it can be seen that the sum linearly increases as the number of terms increases, \emph{the limit cannot be defined}; hence, all \emph{infinite arithmetic series are divergent}.

\paragraph{Arithmetic mean}
The arithmetic mean is the numerical \emph{average} of the elements of a sequence.
It, $mean(\{ a_n\})$ can be expressed as:

\[
  mean(\{ a_n\}) = \dfrac{S_a(n)}{n} = \dfrac{a_1 + a_n}{2} = \dfrac{2a_1 + d(n-1)}{2}
\]

where the first part of the statement is the definition of the mean, the second and third parts came from the definition of the arithmetic series.
From the above statements, it can be said that the arithmetic mean, more informally ``the midle term", of an arithmetic sequence is equal to the arithmetic mean of the first and last terms.

\paragraph{Geometric sequences}
Geometric sequences are sequences of real numbers wherein $\exists d \in \real g_n = r\cdot g_{n-1}$.
This means that consecutive elements of the sequence are separated by a \emph{common ratio} $r$.
The second example given above is an arithmetic sequence with $r = -2$ and $n = 6$.
If $abs{r} > 1$, the absolute value of every succeeding $g_n$ is increasing, else if $abs{r} < 1$, the absolute value of every succeeding $g_n$ is decreasing.

\paragraph{Finite geometric series}
The finite geometric series can be expressed as follows:

\[
  S(n) = a_0 \cdot \dfrac{1-r^n}{1-r}  
\]

where $a_0$ is the first term of the series, $r$ is the common ratio, and $n$ is the number of terms.

\paragraph{Infinite geometric series}
The series sum can expressed as follows:

\[
  S_g(n) = a_0 \cdot \dfrac{1}{1-r} = \lim_{n\to \infty} a_0\cdot\dfrac{1-r^n}{1-r}
\]

where the variable definitions still hold.
Take note that the limit can only be evaluated when $abs{r} < 1$ (\emph{convergent}).
Otherwise, the limit cannot be evaluated, DNE, when $abs{r} > 1$ (\emph{divergent}).

\paragraph{Geometric mean}
The geometric mean of a sequence is defined as:

\[
  gmean(\{ g_n\}) = \sqrt[n]{S_g(n)}
\]

Similar with the arithmetic mean, the geometric mean, also informally ``the middle term'' can also be expressed as the geometric mean of the first and last terms, or any two terms of the same ``distance'' from the middle term.

\subsubsection{Various sequences}
There are also many ways to define sequences.
Presented in this subsection are common sequences and series.

\paragraph{Fibonacci sequence}
One of the more popular sequences known is the Fibonacci sequence.
It is a basic sequence defined recursively from the sum of two previous terms.
The third example at the beginning of this section is an example of a Fibonacci sequence starting from $a_0 = 1, a_1 = 1$ and $n = 9$.

\paragraph{Triangular number sequence}
The triangular number sequence $\{ t_n\}$, is a sequence of numbers wherein each term $\{t_k\}, k \in \{ 1, 2, 3, \ldots, n\}$ is the sum of the arithmetic series from $1$ to $n$.
The fourth example at the beginning of this section is an example of a triangular number sequence of length $7$.

\paragraph{Telescoping series}
A telescoping series is a series wherein the partial sums of the expression yield \emph{fewer} terms due to cancellation of \emph{equivalent} addends.
An example of such a series is:

\[
  \sum_{n=1}^{N} \dfrac{1}{n^2+n} = 1 - \frac{1}{n}
\]

\subsection{Recurrence formulae and mathematical induction}
Sequences can also be defined, like the Fibonacci sequence, recursively through \textbf{recursive formulae}.
Recursion is defined as an operation in repetition with respect to previously calculated elements.

\paragraph{Recurrence formulae}
Recurrence formula are often useful when in need to define successive terms in a sequence without the need of a general expression.
The recursive formula for the Fibonacci sequence is $f_n = f_{n-1} + f_{n-2}$.
The recursive formula for an arithmetic sequence is $a_n = a_{n-1} + d$.
The recursive formula for a geometric sequence is $g_n = r\cdot g_{n-1}$.
The recursive formula for a triangular number series is $t_n = t_{n-1} + n$.
Usually, simple sequences usually have simpler recursive formulae compared to their general counterparts.
As a consequence, they are \emph{simpler} to evaluate \emph{in succession}.

\paragraph{Principle of mathematical induction}
The principle of mathematical induction is a method of proof that relies on assumptions on the recursive properties of expressions that can apply to all integers.
In this paragraph, we will attempt to prove $\forall\ n \in \natl$ that $f(n) = n^3 - n$ is divisible by $3$ using mathematical induction.
There are four basic steps in proving:

\subparagraph{Proof on a trivial cases}
Show that the statement works on easy cases.
For $n=1$, $f(1) = 0 \equiv 0(\textrm{mod} 3)$, and $n=2$, $f(2) = 6 \equiv 0 (\textrm{mod} 3)$.
This step ensures that there are cases wherein the statement holds.

\subparagraph{Assumption on $k \in \natl$}
Assume that the statement holds for some $n \in \natl$.
For some $n = k$, $f(k) = k^3 - k \equiv 0(mod 3)$.
Adding $3k^2 + 3k$ to both sides gives:

\begin{align*}
  k^3 + 3k^2 + 3k - k &\equiv 3k^2 + 3k (mod 3) \equiv 0 (mod 3) \\
  k^3 + 3k^2 + 3k + 1 - k - 1 &\equiv 0(mod 3) \\
  (k+1)^3 - (k+1) = f(k+1) &\equiv 0(mod 3)
\end{align*}

\subparagraph{Conclusion}
Hence, we have shown that if the statement holds for $n = k \in \natl$, then it must hold for $n = k+1$.
Since $\natl$ is closed under addition, the statement must hold $\forall\ n \in \natl$.
