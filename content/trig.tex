\subsection{General angles}

The unit circle is a standard figure in which trigonometric identities are derived from.
The unit circle is a circle of \emph{radius} of $1$ centered at the \emph{origin} $(0, 0)$.
An angle in standard position is described as an angle a \emph{vector} makes with respect to the positive $x$-axis measured \emph{counter-clockwise}.
For any general angle $\theta$, the basic trigonometric functions that can be applied are:

\begin{itemize}
  \item \sin(x)
  \item \cos(x)
  \item \tan(x)
\end{itemize}

\subsection{Trigonometric functions and their basic properties}

\paragraph{Mapping}

\paragraph{Concavity/Shape}

\paragraph{Periodicity}

\subsection{Trigonometric functions and their graphs}

\paragraph{Properties and graphs}

\subsection{Addition theorems for trigonometric functions}

\subsubsection{Sum to product relations}

\subsubsection{Product to relations}

\subsection{Applications of the addition theorems}
