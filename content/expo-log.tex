\subsection{Exponential functions}

Exponential functions are mappings that involve exponentiations of expressions with independent variables with respect to constants.
Such expressions are \emph{transcendental} in nature; hence, they do not follow simple algebraic relations and operations as will be pointed out later.
These functions are injective, \emph{the domain and follow a one-to-one relation}
Examples of these expressions are the following:
\begin{enumerate}
  \item $f(x) = 2^x$
  \item $g(k) = 4^{k+2}$
  \item $h(x) = e^{3x^2+2}$
\end{enumerate}

\subsubsection{Expansion of exponents}

Exponentiation can be defined as the recursive multiplication of \emph{base} expressions due to multiplicities of an \emph{exponent}.
The expression $x^3$ is equivalent to $x\cdot x\cdot x$. Here, $x$ is multiplied thrice to itself.

Multiple expressions can be made from these properties of exponents.
The properties can be described as the laws of exponents as shown by the table below.

\begin{table}[h!]
  \centering
  \caption{Laws of Exponents}

  \begin{tabular}{c|c}
    $x^0 = 1; \forall\ x \neq 0$ & $x^1 = x$ \\
    $x^a \cdot x^b = x^{a+b}$ & $\frac{x^a}{x^b} = x^{a-b}$ \\
    $x^-a = \frac{1}{x^a}$ & $x^{\frac{1}{a}} = \sqrt[a]{x}$ \\
    $(x^a)^b = x^{ab}$ & $a^x \cdot b^x = (ab)^x$
  \end{tabular}
\end{table}

\subsubsection{Exponential functions and their graphs}

\paragraph{Injectivity} As noted above, the exponential functions are \emph{injective}.
This means that there will only be \emph{one} value in the range for every value in the domain.
Generally, singular exponential functions are injective.
Combinations or superpositions (from addition or subraction) can be \emph{unpredictable}, but their behaviors can be shown by calculating the derivatives of the said functions.

\paragraph{Span} Singular exponential expressions span half of $\cartes$.
They have horizontal asymptotes, and depending on the sign of the coefficient is the direction of the graph.
The horizontal asymptotes can be calculated by finding the limits at infinity ($-\inf$ or $\inf$).
This means that the range of simple exponential functions are restricted to half of $\real$, while the domain spans $\real$ in general.

\paragraph{Concavity} It can be noted that the derivatives of exponential functions are also exponential.
The term exponential \emph{growth} refers to the nature of the rate of change (derivative) is very similar to the expression of the represented function:
$$\frac{d}{dx} e^x = e^x$$
$$\frac{d}{dx} -e^{x} = -e^x$$
If the coefficient is positive, as in $3^x = f(x)$, the graph is convex (concave up) and increasing towards $-\inf$ (shown by calculating the limits at$\inf$).
If the coefficient is negative, as in $-5^{x+2} = g(x)$, the graph is concave (concave down) and decreasing towards $-\inf$.

\subsection{Logarithmic functions}

Logarithmic functions are mappings that involve logarithmic expressions, which are defined as inverses of exponential expressions.
The logarithm can be defined as:
$$\log_a b = c$$
where
$$a^c = b$$
Examples of logarithmic functions are:
\begin{enumerate}
  \item $f(x) = \log_3 x$
  \item $g(k) = \log_e (x+2) = \ln (x+2)$
  \item $h(q) = \log_10 x = log x$
\end{enumerate}

Similar to exponential functions, they are also transcendental and follow the same rules as their inverses.
The laws of exponents can be translated to the logarithmic counterparts as follows

\subsubsection{Properties of logarithms}

\begin{table}
  \centering
  \caption{Laws on Logarithms}

  \begin{tabular}{c|c}[h!]
    $\log_a 1 = 0 \forall\ a \neq 0$ & $\log_a a = 1$ \\
    $\log_a x + \log_a y = \log_a (xy)$ & $\log_a x - log_a y = log_a (\frac{x}{y})$ \\
    $-\log_a x = \log_a \frac{1}{x}$ & $c\log_a b = \log_a (b^c)$ \\
    $\frac{log_a x}{log_a y} = \log_y x$
  \end{tabular}

\end{table}

\subsubsection{Logarithmic functions and their graphs}

Logarithmic functions are, generally, the inverses of exponential functions.
This means that the graphs of simple logarithmic functions are \emph{reflections} of their respective inverses with respect to $f(x) = x$.

\paragraph{Injectivity} Simple logarithmic functions are also injective.
However, their superpositions are not linear (unpredictable).
Derivatives are useful tools for the general description for function behavior.

\paragraph{Span} They also span half of $\cartes$.
Since they are diagonal reflections, the horizontal asymptotes of corresponding exponential functions will be numerically be equivalent to the vertical asymptotes.
Similarly, the domain is restricted to half of $\real$, while the range spans $\real$.

\paragraph{Concavity}
The second derivative of a logarithmic expression is:

$$\frac{d^2}{dx^2} \log_a (x) = -\frac{1}{x^2 \cdot \ln{a}}$$

which is always \emph{negative} $\forall\ x > 1$ but \emph{positive} $\forall 0 < x < 1$.

Hence, the graph is \emph{concave} $\forall\ x > 1$ ang \emph{convex} $\forall\ 0 < x < 1$.
