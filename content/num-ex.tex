\subsection{Numbers and Sets}
\subsubsection{Logic}
Mathematical logic is an important part in analysis.
Presented as follows are some relevant concepts. 

\begin{table}[h!]
    \centering
    \caption{Common logic terms and definitions}
    
    \begin{tabular}{c|c|c}
        conjunction & $p \land q$ & $p$ and $q$ \\
        disjunction & $p \lor q$ & $p$ or $q$ \\
        statement & $p \Longrightarrow q$ & ``$p$ implies $q$" \\
        converse & $q \longrightarrow p$ & ``$q$ implies $p$" \\
        equivalence & $p \iff q$ & ``$p$ if and only if $q$"
    \end{tabular}
\end{table}

\begin{table}[h!]
    \centering
    \caption{``Tautologies in propositional calculus"}
    \begin{tabular}{c|c}
        $p \lor \neg p$ & ``Law of excluded middle" \\
        $\neg (p \land \neg p)$ & ``Law of contradiction" \\
        $\neg (\neg p) \iff p$ & ``Negation of negation" \\
        $p \Longrightarrow q \iff (\neg p \lor q)$ & ``Sufficiency" \\
        $q \Longrightarrow p \iff (\neg q \lor p)$ & ``Necessity" \\
        $\neg (p \land q) \iff \neg p \lor \neg q$ & De Morgan's Law \\
        $\neg (p \lor q) \iff \neg p \land \neg q$ & De Morgan's Law \\
        $(p \Longrightarrow q) \iff (\neg q \Longrightarrow \neg p)$ & ``Law of contraposition / the contrapositive" \\
        $[(p \Longrightarrow q) \land (q \Longrightarrow r)] \Longrightarrow (p \Longrightarrow r)$ & ``Law of Transitivity" \\
        $p \land (p \Longrightarrow q) \Longrightarrow q$ & ``\textit{Modus ponens} or Law of detachment" \\
        $q \land (\neg p \Longrightarrow \neg q) \Longrightarrow p$ & ``Indirect proof" \\
        $(p_1 \lor p_2) \land (p_1 \Longrightarrow q) \land (p_2 \Longrightarrow q) \Longrightarrow q$ & ``Distinction of cases"
    \end{tabular}
\end{table}

Table 2 is explained below:
\begin{enumerate}
    \item Either a statement or its negation is true.
    \item A statement and its negation cannot be true at the same time.
    \item The negation of the negation of the statement is the statement.
    \item {\bfseries Sufficiency}. A premise is \emph{sufficient} for a conclusion iff the conclusion is true whenever the premise holds.
    \item {\bfseries Necessity}. A conclusion is \emph{necessary} for a premise iff the premise is true whenever the conclusion holds.
    \item If the conjunction of two statements is false, then at least one of them is false.
    \item If the disjunction of two statements is false, then both of them are false.
    \item A premise is sufficient for a conclusion if and only if the negation of the conclusion is sufficient to show that the premise is false.
    \item Logic is transitive.
    \item If a statement is true and it is sufficient for a conclusion then the conclusion holds.
    \item If a conclusion holds and it is necessary for a premise then the premise holds.
    \item If at least one premise (case) that is sufficient for a conclusion holds, then it is necessary to conclude.
\end{enumerate}

\subsubsection{Sets}
\paragraph{Sets}
Sets, in general, are collections of well-defined objects, or even sets!
Sets can have \emph{any whole number} of elements, called cardinality.
A set with only one element is called a \emph{singleton}, such as $\{\varnothing\}$.
A set with no elements is called an \emph{empty set or null set}, $\varnothing$.

The elements of sets should be unique and well defined. Sets can be defined in different ways:
\begin{enumerate}
    \item ``Let $S$ be the set of even integers"
    \item $S = \{0, 1, 2, 3\}$---Roster notation
    \item $S = \{x : x \neq 1\}$---Set-builder notation
\end{enumerate}

\paragraph{Subsets}
There are also collections within collections---subsets---denoted as $M \subset N$.
Here, we have a subset $M$ contained within $N$.
This follows that all elements of $M$ can be found in $N$.

\paragraph{Set operations}
Common sets operations include \emph{set intersection}, $\cap$, and \emph{set union}, $\cup$.
The intersection between two well defined sets is the collection of elements present in \emph{both} sets.
The union of two well-defined sets is the collection of elements that can be found in \emph{either} set.
Other set operations include the set difference $\setminus$ and the set complement $'$. The set difference $M \setminus N = \{x | x \in M \land x \notin N\}$. The complement $M' = \Omega \setminus M$, where $\Omega$ is a universal set.

\subsubsection{Real Numbers}
In beginning algebra, most of the operations apply to the real number system $\mathbb{R}$.
This number system is then divided into two sets---the set of rational numbers $\mathbb{Q}$ and the set of irrational numbers $\mathbb{Q}'$.

\paragraph{Rational Numbers}
Rational numbers are numbers that can be expressed as $\frac{p}{q}$ where $p, q$ are integral.
This means that these numbers \emph{can be expressed as fractions}.
Examples of rational numbers are $\frac{1}{3}$, $\frac{134}{235}$, and $\frac{3}{1}$.
The set of irrational numbers is the complement of $\mathbb{Q}$ in $\mathbb{R}$.
These numbers, hence, \emph{cannot be expressed as fractions}.
Examples of irrational numbers are $e$, $\pi$, $\sqrt{2}$.

\paragraph{Integers}
$\mathbb{Q}$ is subdivided into the set of integers $\mathbb{Z}$ and the set of non-integers.
Subsets of $\mathbb{Z}$ the set of whole numbers, $\mathbb{W} = \{0, 1, 2, 3, \ldots\}$, and the set of natural numbers $\mathbb{N} = \{1, 2, 3, \ldots\}$.

More subsets can be defined, by parity, such as the even number set and the odd number sets within $\mathbb{N}$. There is also a special subset or natural numbers called the set of prime numbers. This set will be described in some detail in a later section.

\subsubsection{Propositions}
Properties of subsets:
\begin{enumerate}
    \item Reflexivity -- ``A set is a subset of itself"
    \item Transitivity -- ``$M \subset N \land N \subset P \Longrightarrow M \subset P$"
    \item $\varnothing \subset M\ \forall\ M$
\end{enumerate}

These properties imply that:
\begin{enumerate}
    \item The number of subsets of any set $M$ is $2^{N(M)}$, where $N(M)$ is the cardinality of $M$.
    \item The number of proper subsets of $M$ is $2^{N(M)}-1$
\end{enumerate}

\paragraph{De Morgan's Law}
The law applies to the set operations $', \cap, \cup$.
The propositions are stated as follows:

\begin{enumerate}
    \item $(M \cup N)' = M' \cap N'$
    \item $(M \cap N)' = M' \cup N'$
\end{enumerate}

\paragraph{Cartesian Product}
The Cartesian Product of two sets $X$ and $Y$, $X \times Y = \{(x, y) | x \in X \land y \in Y\}$. This means that $\mathbb{R}^2$ is the two-dimensional Cartesian plane, $\mathbb{R}^3$ is the three-dimensional Cartesian plane, and so on.

\subsection{Calculation of Expressions}
Expressions, in mathematics, are segments of statements.
These expressions often involve operations on quantities.
Examples are as follows:

\begin{enumerate}
    \item $1 + 1$---Constant expression
    \item $x$---Monomial
    \item $(3x - 2)^2$---Polynomial
    \item $\frac{3x+2}{4x+3}$---Rational expression
    \item $(x+y)^4$---Expression in multiple variables
\end{enumerate}

\subsubsection{Expansion and Factorization of Expressions}

\paragraph{FOIL method} For multiplication of binomials, $(3x+2)\cdot(4x+3)$, the FOIL method for expansion provides an equivalent of the expression $(3x+2)(4x+3) = 12x^2+17x+6$. 
The FOIL method is an application of the \emph{distributive} property of multiplication over addition; however this is limited to multiplying binomials.

\paragraph{Rainbow method} For multiplying more complex polynomials, the rainbow method is a generalized version of the FOIL method. $(x^2+2x+1)(x+1) = x^3 + 2x^2 + x^2 + 2x + x + 1 = x^3 + 3x^2 + 3x + 1$. Again, this is also derived from the distributive property of multiplication over addition.

\paragraph{Regrouping} For factoring expressions, $a^2 + 2ab + b^2$, regrouping terms with common factors is reasonable so that common terms will be factored out: $a^2 + ab + b^2 + ab = a(a+b) + b(a+b) = (a+b)^2$.
This is also a consequence of the distributive property.
In order to properly factor expressions, there must be ways to redistribute the terms.

\paragraph{Commonly used expressions}
Some of the commonly used expressions are listed below:

\begin{enumerate}
    \item Square of a binomial: $(x+a)^2 = x^2+a^2+2ax$
    \item Square of a trinomial: $(x+a+b)^2 = x^2+a^2+b^2+2ax+2bx+2ab$
    \item Difference of two squares: $x^2 - a^2 = (x-a)(x+a)$
    \item Difference of two cubes: $x^3 - a^3 = (x-a)(x^2+ax+a^2)$
    \item Sum of two cubes: $x^3+a^3 = (x+a)(x^2-ax+a^2)$
    \item Difference of powers of five: $x^5-a^5 = (x-a)(x^4+ax^3+a^2x^2+a^3x+a^4)$
    \item Sum of powers of five: $x^5+a^5 = (x+a)(x^4-ax^3+a^2x^2-a^3x+a^4)$
\end{enumerate}

\subsubsection{Linear inequalities}
There are two kinds of inequalities: \emph{strict} and \emph{non-strict} inequalities.
Strict inequalities are statements of non-equality providing no solutions for equality.
Non-strict inequalities, on the other hand, provide solutions for equality.

Examples of inequalities are the following.
Also, note that the statements are complements ($-ve$s) of each other.

\begin{enumerate}
    \item $x+3 \leq 4$---Non-strict inequality
    \item $x+3 > 4$---Strict inequality
\end{enumerate}

Solving linear inequalities is similar to solving linear equations.

\subsubsection{Equations and inequalities with $\abs{x}$}
\paragraph{Absolute value function $\abs{x}$}
This function gives the distance of the value of the expression from $0$.
Since it returns a distance, it should be \emph{non-negative}.
Formally the absolute value is defined as:
\[
\abs{x} = 
\begin{cases}
x & x \geq 0 \\
-x & x < 0
\end{cases}
\]
Examples are shown below:
\begin{enumerate}
    \item $\abs{3} = 3$
    \item $\abs{-10} = 10$
\end{enumerate}

\emph{There are two cases for the value of a real-valued expression: it is non-negative or negative}. Thus, both cases must be considered.

\paragraph{Equations with $\abs{x}$}
Since the absolute value of a number is its distance from $0$ in the real number system, there are two solutions to equations of absolute values of linear equations.
If $\abs{x+a} = 10 \iff x+a=10 \lor x+a=-10$.

\paragraph{Inequalities with $\abs{x}$}
Inequalities containing absolute values are slightly different from equations.
Since absolute values are non-negative, then the inequality $\abs{13-x} < 0$ would not make sense. 
What makes absolute value inequalities different is the set of cases used for solving.

Shown are the four cases and their logical equivalents:
\begin{enumerate}
    \item $\abs{x+3} \leq 4 \iff -4 \leq x+3 \leq 4$
    \item $\abs{x+3} < 4 \iff -4 < x+3 < 4$
    \item $\abs{x+3} \geq 4 \iff (-4 \geq x+3) \lor (4 \leq x+3)$
    \item $\abs{x+3} > 4 \iff (-4 > x+3) \lor (4 < x+3)$
\end{enumerate}

\paragraph{Triangle inequality (Algebraic)}
This is a famous inequality involving absolute values.
This will be restated in a later section (see section \ref{figs}).

\[
\abs{x+y} \leq \abs{x} + \abs{y}
\]
