\subsection{Expressions and proofs}

\subsubsection{Division of polynomials, rational expressions, and binomial theorem identities}
\paragraph{Euclidean division of polynomials}
The \emph{Euclidean division}, or simply division, of polynomials is the counterpart of the Euclidean division of integers to polynomials.
A polynomial $P(x)$ can be expressed as:
$$P(x) = Q(x)\cdot D(x) + R(x)$$
where $D(x)$ is the divisor, $Q(x)$ is the \emph{quotient}, and $R(x)$ is the remainder upon division.
As examples:
\begin{enumerate}
  \item $x^2+3x+2 = (x+1)(x+2) + 0$
  \item $x^2 = (x-1)(x+1) + 1$
\end{enumerate}

This division of polynomials can be carried out using \emph{long divison} and \emph{synthetic division}.
Application of these techniques are \textbf{not} covered in this reviewer.

\paragraph{Long Division}
As the name suggests, this is the method of division that shows all of the parts of the polynomials being divided.
This is useful for finding shorter techniques for dividing polynomials.
The technique of synthetic division is derived from the application of long division.

\paragraph{Synthetic Division}
This division technique is derived from the long division of polynomials.
This makes use only of the coefficients of the terms of the dividend and the divisor.
This is normally used when dividing polynomials of a single term to a linear polynomial of the same term, \textit{eg.} $3x^3 - 4x + 5$ divided by $x+2$.
Extensions to quadratic forms have been made and further generalizations can also be made.

\paragraph{Rational Expressions}
Rational expressions are expressions involving \emph{ratios} of polynomials.
There are two kinds of rational expressions---\emph{proper} and \emph{improper} rational expressions.
A \textbf{proper} rational expression is a ratio between a polynomial of a higher degree, or the same, to another polynomial.
An \textbf{improper} rational expression is a ratio between a polynomial of a lower degree to another polynomial.
Examples of rational expressions are:

\begin{enumerate}
  \item $\frac{x^2+x+1}{x^3+4x+2}$---a proper rational expression
  \item $\frac{x^3+4x+2}{x^2+x+1}$---an improper rational expression
\end{enumerate}

Since improper rational expressions is a division operation on two polynomials, it is necessary that they can be expressed in the form $Q(x)\cdot D(x) + R(x)$ by Euclidean division.
From expression:

\begin{align*}
  P(x) &= Q(x)\cdot D(x) + R(x) \\
  \dfrac{P(x)}{D(x)} &= Q(x) + \dfrac{R(x)}{D(x)}
\end{align*}

Here, the $LHS$ is the \emph{improper} rational expression, $Q(x)$ is the polynomial quotient, and the last term is a \emph{proper} rational fraction, the remainder.

\subsubsection{Proofs of equations and inequalities}
This subsection will contain important equations and inequalities used for proving.
Important equations for proving algebraic equalities are shown in section \ref{num-ex}.

\paragraph{QM-AM-GM-HM inequalities}
This family of inequalities is famous for applications in proving related inequalities.
This is a relation between the \emph{quadratic mean} (QM), \emph{arithmetic mean} (AM), \emph{geometric mean} (GM), and \emph{harmonic mean} (HM) of \emph{positive real numbers}.
Equality is achieved when \emph{all} numbers in the set are \emph{equal}
The relation is given as follows:

\begin{align*}
  QM &\geq AM &\geq GM &\geq HM \\ \\
  \sqrt{\dfrac{1}{N}\cdot \sum\limits_{n=1}^{N} a_n^2} &\geq \dfrac{1}{N}\cdot \sum\limits_{n=1}^{N} a_n &\geq \Big(\prod_{n=1}^{N}a_n \Big)^{\frac{1}{N}} &\geq \Big(\dfrac{1}{N}\cdot \sum\limits_{n=1}^{N}\frac{1}{a_n}\Big)^{-1}
\end{align*}

\subparagraph{Tips}
Use these inequalities when the expressions appear \emph{cyclic}.
This means that terms have repeating ``patterns''.

\paragraph{Cauchy-Schwarz Inequality}
This inequality is also used in proving some related inequalities.
These state that the product of a sum of squares is grater than or equal to the sum of the square of the sum of the products of the terms.
Equality holds when $\frac{a_n}{b_n}\ \forall\ n\leq N$.
Symbollically,

$$
(a_1^2+a_2^2+a_3^2+\ldots +a_n^2)(b_1^2+b_2^2+b_3^2+\ldots +b_n^2) \geq (a_1b_1+a_2b_2+a_3b_3+\ldots +a_nb_n)^2 
$$

\subparagraph{Tips}
Use this inequality when comparing inequalities involving \emph{both squares and their products}.
Applications may not be obvious.

\subsection{Equations of higher degree}
Equations of degree $3$ and above do not follow quadratic, and other simple, relations and are most of the time harder to solve.
Completing the cube (for degree 3 polynomials) and so forth for higher degree polynomialsare very tedious to do.

\subsubsection{Complex numbers and solutions of quadratic equations}
The set of complex numbers $\cplx$ contains the set of real numbers $\real$ and the set of non-real numbers $\real '$.
Complex numbers have imaginary components and can be expressed in the form $a + bi$ where $a, b \in \real$ and $i$ is the imaginary unit ($i = \sqrt{-1}$).

As stated in section \ref{quad}, quadratic expressions with $D < 0$ have no real roots but they have \emph{two non-real, complex roots}.
Given that $\sqrt{-1} = i$, it is then possible to calculate the complex roots of quadratic functions.

\subsubsection{Factor theorem}
This theorem states that if $P(a) = 0$, then $x-a$ is a factor of $P(x)$ \textit{ie. it leaves a remainder of $0$ upon division}.
This theorem can be shown as a consequence of Euclidean division.

\subsubsection{Properties of higher degree polynomials and methods of solving them}
\paragraph{Rational root theorem}
This theorem states that given a polynomial $P(x) = a_nx^n + a_{n-1}x^n-1 + a_{n-2}x^{n-2} +\ldots +a_0$, the set $S$:
$$ S = {\dfrac{b_0}{b_n} | b_0\mid a_0; b_n\mid a_n} $$
contains all possible rational roots of the equation.
