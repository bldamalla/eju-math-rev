\subsection{Lines and Circles}

\subsubsection{Coordinates of a Point}
Considering $cartes$, there are two parts to the coordiantes of a point.
The first part is the x-coordinate known as the \emph{abscissa} of the point.
The second part is the y-coordinate known as the \emph{ordinate} of the point.
Using the abscissa and ordinate of a point, it is possible to plot them onto $cartes$.

\paragraph{The Four Quadrants}
The Cartesian plane is divided into $4$ regions as quadrants.

\subsubsection{Equations of Lines}
A \emph{line} is a collection of points that extend in both directions.
Lines can be represented in $real ^2$ as relations between $x$ and $y$.
Examples of these relations are (equations are the statements that must be satisfied):
\begin{enumerate}
  \item $L_1 = \{(x, y) | y = x\}$
  \item $L_2 = \{(x, y) | y = -x\}$
  \item $L_3 = \{(x, y) | y = 0\}$
  \item $L_4 = \{(x, y) | x = 0\}$
\end{enumerate}

Items 1 and 2 above show a relation that exists in both $x$ and $y$.
When all points following this relation (Item 1) are plotted, a line passing through the origin at $45\degg$ will be seen.
When all points following the second relation (Item 2) are plotted, a line passing through the origin $-45\degg$ will be seen.
Items 3 and 4 show relations that depend only on one of $x$ and $y$.
Item 3 presents a horizontal line passing through the origin---also called the \emph{$x$-axis}.
Item 4 presents a vertical line passing through the origin---also called the \emph{$y$-axis}.

\paragraph{Intercepts}
Intercepts are the values at which the line crosses an axis.
The $x$ and $y$ intercepts---$a$ and $b$---of Items 1 and 2 above are $0$ and $0$, respectively.

\paragraph{Slope}
The slope of a line is a measure of ``steepness''.
Its value is calculated as the \emph{first derivative} of the equation as a function or ``rise over run''.

\paragraph{Forms of Equations of Lines}
Equations of lines can be expressed in different forms such as the \emph{general form} and the \emph{standard forms}.

\subparagraph{General Form}
This form is easily distinguishable since \emph{one side of the equation is $0$}. 
Examples are $x \pm y = 0$, $x = 0$, $y = 0$.

\subparagraph{Standard Forms}
There are many different standard forms of equations of lines.
These forms show the different properties of the lines they define.
Listed below are some of them and the properties they give.

\begin{table}[h!]
  \centering
  \caption{Standard forms and properties of lines}
  \begin{tabular}{c|c|c}
    Slope-intercept form & $y = mx + b$ & Slope and $y$-intercept of the line \\
    Point-slope form & $(y-y_1) = m(x-x_1)$ & Slope and a point $(x_1, y_1)$ on the line \\
    Two point form & $(y-y_{1/2}) = \big(\frac{y_2-y_1}{x_2-x_1}\big)\big(x-x_{1/2}\big)$ & Two points on the line \\
    Two-intercept form & $\frac{x}{a} + \frac{y}{b}=1$ & \emph{Non-zero} intercepts of the line
  \end{tabular}
\end{table}

\subsubsection{Equations of Circles}

\paragraph{Distance formula}
This formula is based on the Pythagorean theorem and it is used to get distances between points.
$$D = \sqrt{(x_2-x_1)^2 + (y_2-y_1)^2}$$

As said in section \ref{figs}, a circle is a collection of points \emph{equidistant} to a fixed point called the \emph{center}.
Using the distance formula, we can find a relation wherein the distance of points to a ``center'' is \emph{constant}---a circle.

$$(x-h)^2 + (y-k)^2 = r^2$$

Given above is the standard form of the equation of a circle.
Like lines, a general form ($RHS = 0$) for the equation can also be expressed.

\subsubsection{Relative Positions of a Circle and a Line}

Lines can be drawn alongside a circle and relative positions can be determined.
Two of the relative positions were already stated in section \ref{figs}---tangent and secant.
In this section, we define a third relative position called the \textbf{external line}.
As the name suggests, an external line is a line that does not intersect the circle.

To determine the relative position of a line with respect to a circle, the following system should be solved.

$$\begin{cases}
  (x-h)^2 + (y-k)^2 &= r^2 \\
  y - mx &= b
\end{cases}$$

Solving the system means finding points that satisfy \emph{both} equations.
This then means that such points can be found on both the line and the circle.
If there are \emph{two} solutions, then the line is a \emph{secant} to the circle.
If there is \emph{only one} solution, then the line is \emph{tangent} to the circle.
Lastly, if there are no solutions, then the line is an \emph{external line} to the circle.

\subsection{Locus and Region}

\subsubsection{Locus defined by an equality}

A \textbf{locus} is a set of points that satisfies a given equality.
Loci (relations) are visually represented in $\cartes$.
As examples related to this section, a circle is a locus equdistant to a center, a line is a locus of constant variation.

\subsubsection{Region defined by an equality}

Loci, whether closed or not closed, divide $\cartes$ into two areas or \textbf{regions} of inequality.
Regions above the locus satisfy a standard $\geq$ relation while regions below the locus satisfy a standard $\leq$ relation.
