\subsection{Number of possible outcomes}
\paragraph{Events}
Events can be broken down into sets of events either \emph{dependent} or \emph{independent} of each other.
\subparagraph{Independent events}
These are events wherein the outcome of an event does not rely on another.
An example of such events are ``picking an ace" and ``picking a red card" from a standard deck.
\subparagraph{Dependent events}
These are events wherein the outcome of an event relies on the outcome of a first event.
An example of such events are first ``picking a black card", and throwing it away then ``picking a `Queen"\!".

\subsubsection{Principles of counting}
Cardinality refers to the number of elements in a set.
Such sets---also called the \emph{outcome space} can be the sets of possible outcomes for independent events.

\subsubsection{Permutations and combinations}

\subsection{Probability and its fundamental properties}

\subsection{Independent trials and probability}

\subsection{Conditional probability}

