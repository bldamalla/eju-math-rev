\subsection{Quadratic Functions and Their Graphs}
\paragraph{Relations and Functions}
A relation is a mapping between two well-defined sets $M$ and $N$.
Operating on real-valued sets, every relation $S \subset \mathbb{R}^2$ and can be graphed in a two-dimensional Cartesian plane.
There are different kinds of relations depending on the number of associations of elements of each set in another.
Of such relations there are only some that are considered functions.
Functions are operations on expressions that map values to another set of values.
Hence, if a value or expression in the \textbf{domain} is mapped to multiple elements in a certain \textbf{codomain} set the relation is not a function.

\subparagraph{Vertical and Horizontal line test}
The vertical line test is a method to determine whether a relation is a function.
The validity of this test arises from the definition of a function.
The horizontal line test is a method to determine whether a function is injective or non-injective.

The four kinds of relations are as follows:

\subparagraph{One-to-one relations/injective}
In this kind of relation, an element in set $N$ is associated with \emph{at most one} element in $M$.
Given $M = {3, 5, 6}$ and $N = {6, 10, 12}$.
A one-to-one relation for the given sets is given as ``the elements of $N$ are twice the elements of $M$".
Equivalently, this can be expressed as $N = {2x | \forall x \in M}$.
Another one-to-one relation for the given sets is the relation which defines the set $S = {(3, 12), (5, 10), (6, 6)}$.
This relation is a function since there is exactly one element mapped to each element in $M$ in $N$.
This function is injective since at most one element in $M$ is mapped to every element in $N$.

\subparagraph{One-to-many relations}
In this kind of relation, an element in set $M$ is associated to more than one element in $N$.
Given $M = {3, 5, 6}$ and $N = {6, 10, 12}$.
A subset of the relation is $S = {(3, 10), (3, 12), (5, 6), (5, 10)}$.
As shown $3$ and $5$ from $M$ pair are mapped with at least one element in $N$.
This relation cannot be a function since there are more than one values that are mapped to an element in the domain to the codomain.

\subparagraph{Many-to-one relations}
In this kind of relation, more than one element in set $M$, domain, may be associated to \emph{exactly one} element in $N$, codomain.
Given $M = {3, 5, 6}$ and $N = {6, 10, 12}$,
A relation between $M$ and $N$ of this kind is $S = {(3, 6), (5, 6), (6, 10)}$.
As shown there can be more than one mapped element to the codomain from the domain.
This relation is a function since there is only one element from $N$ that is mapped to every element in $M$.

\subparagraph{Many-to-many relations}
In this kind of relation, more than one element in $M$ may be associated to an element in $N$ and vice versa.
This cannot be a function, by definition.

\paragraph{Quadratic expressions and functions}
Quadratic expressions are \emph{polynomial expressions of degree $2$}.
Examples of quadratic expressions are $x^2$ and $x^2+6x+9$.
Quadratic functions are operations on a domain, $\mathbb{R}$ that is defined with mapping with quadratic relations.
An example is $f(x) = x^2; f(3) = 9$

\subsubsection{Variation in values of quadratic functions}
% I don't get this honestly
Quadratic functions vary with respect to the square and of the value of the operand.
Hence the variation varies linearly with respect to the operand---via getting the derivative (see section \ref{diffcalc}) of the function.
Values of quadratic functions either increase then decrease or decrease then increase.
Geometrically, the \emph{locus} (set of points) defining the relation is a \emph{vertical parabola}.
This then means that quadratic functions have maximum and minimum values.

\subsubsection{Maximum and minimum values of quadratic functions}
A quadratic function, being parabolic, is either concave up or concave down.
Depending on its orientation, a parabola can have either a minimum value or a maximum value.
If the function is led with a negative coefficient, then it is concave down.
Otherwise, it is concave up.

It can be shown by the derivative that \emph{concave down quadratic functions have local maxima} and \emph{concave up quadratic functions have local minima}.
Not surprisingly, since the concavity is a constant function of the operand---via the second derivative test (see section \ref{diffcalc})---\emph{local extrema are also absolute extrema}.

For a given quadratic function $f(x) = ax^2+bx+c$, by the first derivative test (see section \ref{diffcalc}), the local extremum can be found at $x = \frac{-b}{2a}$.
The value of the extremum, by substitution is $y = \frac{4ac-b^2}{4a^2}$.

\subsubsection{Determining quadratic functions}
\paragraph{Connect the dots}
Since functions are operations on some real number $x$ it would be possible to determine functions from points that define it.
As a general rule, to determine a polynomial function of degree $n$ at least $n+1$ points would be needed.
Since quadratic functions are degree $2$ polynomials, there should be about three points.
Given three points, it would then be possible to construct systems of linear equations in $\mathbb{R}^3$.
It would be similar to solving the following matrix:

\[
\begin{bmatrix}
x_1^2 & x_1 & 1 \\
x_2^2 & x_2 & 1 \\
x_3^2 & x_3 & 1
\end{bmatrix}
\cdot
\begin{bmatrix}
a \\
b \\
c
\end{bmatrix}
=
\begin{bmatrix}
f(x_1) \\
f(x_2) \\
f(x_3) \\
\end{bmatrix}
\]

\subparagraph{Given the roots of the function}
This is a special case of ``connect the dots".
A quadratic function has either two real roots or no real roots.
This applies to the first case.
The function can be determined from roots $x_1 and x_2$ as $f(x) = (x-x_1)(x-x_2)$

\subparagraph{Given the y-intercept and vertex}
This is another special case of ``connect the dots".
Here $c$ would be the value of the y-intercept, $h = \frac{-b}{2a}$, and finally $k = \frac{4ac-b^2}{4a^2}$.

\paragraph{Given vertex $(h, k)$ and latus rectum length $4p$}
From the standard form of a vertical parabola, $(x-h)^2 = 4p(y-k)$, it would be easy to determine the function through mapping of the given values.

\paragraph{Given the sum and product of roots}
This also applies if related values can be derived.
Let the sum and product of roots be $s$ and $p$, respectively.
The quadratic function, by Vieta's theorem is given by $f(x) = x^2 + sx + p$.

\subsection{Quadratic Equations and Inequalities}
As a note, \emph{equations and inequalities are not functions; they are statements}.
Functions are \emph{operations} on given operands.
Examples of such statements are:
\begin{enumerate}
    \item $x^2 + 3x + 7 = 5$---a quadratic equation in $x$
    \item $a^4 + 2a^2 + 1 = 0$---a quadratic equation in $a^2$
    \item $x^2 + 3x + 7 \geq 5$---a quadratic inequality in $x$
    \item $a^4 + 2a^2 + 1 < 0$---a quadratic inequality in $a^2$
\end{enumerate}

\subsubsection{Solutions of quadratic equations}
Recall that polynomials that are \emph{factorable} may be reduced to their factors.
Polynomials that cannot be factored are classified as \emph{prime} polynomials.

\paragraph{Factoring}
If the quadratic expressions not prime, the roots (solutions) can be determined from equating the factors to $0$.

\paragraph{Completing the square}
Perfect squares, similar to absolute values, are non-negative.
Then statements such as $x^2 = -1$ and do not make sense in the real number system.
Again, similar to absolute value expressions, cases will be considered in solving statements.
First, transform the statement into one containing a \emph{perfect square trinomial} in one side.
In solving $(x-a)^2 = b$, there are two cases $x-a = \pm \sqrt{b}$.
The solutions follow.

\subparagraph{Quadratic formula}
As a shortcut to completing the square, the quadratic formula was defined from the process.
\[
x = \dfrac{-b \pm \sqrt{b^2-4ac}}{2a}
\]

\subsubsection{Quadratic equations and graphs of quadratic functions}
Quadratic equations are statements with a quadratic expression.
The quadratic expression can be treated as a function of an unknown, and the ``other-hand-side" as the function value, i.e $f(a) = 3a^2 = 27$.
In solving the quadratic equation $f(a)-c = 0$, we find the zeros of a quadratic function $g(a) = f(a)-c$.
The intersections of the graph of $g$ are the roots of the quadratic equation $f(a)-c = 0$.

\subsubsection{Quadratic inequalities and graphs of quadratic functions}
Quadratic functions, such as $f(x)$, divide $\mathbb{R}^2$ into two regions of inequality.
Regions above the graphs of quadratic functions are regions $S = \{(x, y) | y > f(x)\ \forall\ x \in R\}$.
On the other hand, the regions below the functions are regions $T = \{(x, y) | y < f(x)\ \forall\ x \in R\}$.

Similarly from the previous subsection, we can determine the boundaries where the inequality is satisfied and when they are not.

\paragraph{If $g$ is led by a positive coefficient}
This means that the graph of $g$ is concave up.
If $g(a) > 0 \iff f(a) > c$ must be satisfied then the solution set is: $$S = \{x | x < x_1 \lor x > x_2\}$$.
Else if $g(a) < 0 \iff f(a) < c$ must be satisfied then the solution set is: $$S = \{x | x_1 < x < x_2\}$$.
In both cases $x_1$ and $x_2$ are the zeros of $g(a)$ and the roots of $f(a) - c = 0$.

\paragraph{If $g$ is led by a negative coefficient}
This means that the graph of $g$ is concave down.
The whole inequality can be negated for simplicity.
Note that the \emph{inequality symbol must be flipped as well}.
